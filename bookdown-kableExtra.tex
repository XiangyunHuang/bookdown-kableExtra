% Options for packages loaded elsewhere
\PassOptionsToPackage{unicode}{hyperref}
\PassOptionsToPackage{hyphens}{url}
\PassOptionsToPackage{dvipsnames,svgnames*,x11names*}{xcolor}
%
\documentclass[
]{ctexbook}
\usepackage{lmodern}
\usepackage{amssymb,amsmath}
\usepackage{ifxetex,ifluatex}
\ifnum 0\ifxetex 1\fi\ifluatex 1\fi=0 % if pdftex
  \usepackage[T1]{fontenc}
  \usepackage[utf8]{inputenc}
  \usepackage{textcomp} % provide euro and other symbols
\else % if luatex or xetex
  \usepackage{unicode-math}
  \defaultfontfeatures{Scale=MatchLowercase}
  \defaultfontfeatures[\rmfamily]{Ligatures=TeX,Scale=1}
\fi
% Use upquote if available, for straight quotes in verbatim environments
\IfFileExists{upquote.sty}{\usepackage{upquote}}{}
\IfFileExists{microtype.sty}{% use microtype if available
  \usepackage[]{microtype}
  \UseMicrotypeSet[protrusion]{basicmath} % disable protrusion for tt fonts
}{}
\makeatletter
\@ifundefined{KOMAClassName}{% if non-KOMA class
  \IfFileExists{parskip.sty}{%
    \usepackage{parskip}
  }{% else
    \setlength{\parindent}{0pt}
    \setlength{\parskip}{6pt plus 2pt minus 1pt}}
}{% if KOMA class
  \KOMAoptions{parskip=half}}
\makeatother
\usepackage{xcolor}
\IfFileExists{xurl.sty}{\usepackage{xurl}}{} % add URL line breaks if available
\IfFileExists{bookmark.sty}{\usepackage{bookmark}}{\usepackage{hyperref}}
\hypersetup{
  pdftitle={kableExtra 在 bookdown 项目中的使用},
  pdfauthor={黄湘云},
  colorlinks=true,
  linkcolor=Maroon,
  filecolor=Maroon,
  citecolor=Blue,
  urlcolor=Blue,
  pdfcreator={LaTeX via pandoc}}
\urlstyle{same} % disable monospaced font for URLs
\usepackage{color}
\usepackage{fancyvrb}
\newcommand{\VerbBar}{|}
\newcommand{\VERB}{\Verb[commandchars=\\\{\}]}
\DefineVerbatimEnvironment{Highlighting}{Verbatim}{commandchars=\\\{\}}
% Add ',fontsize=\small' for more characters per line
\usepackage{framed}
\definecolor{shadecolor}{RGB}{248,248,248}
\newenvironment{Shaded}{\begin{snugshade}}{\end{snugshade}}
\newcommand{\AlertTok}[1]{\textcolor[rgb]{0.94,0.16,0.16}{#1}}
\newcommand{\AnnotationTok}[1]{\textcolor[rgb]{0.56,0.35,0.01}{\textbf{\textit{#1}}}}
\newcommand{\AttributeTok}[1]{\textcolor[rgb]{0.77,0.63,0.00}{#1}}
\newcommand{\BaseNTok}[1]{\textcolor[rgb]{0.00,0.00,0.81}{#1}}
\newcommand{\BuiltInTok}[1]{#1}
\newcommand{\CharTok}[1]{\textcolor[rgb]{0.31,0.60,0.02}{#1}}
\newcommand{\CommentTok}[1]{\textcolor[rgb]{0.56,0.35,0.01}{\textit{#1}}}
\newcommand{\CommentVarTok}[1]{\textcolor[rgb]{0.56,0.35,0.01}{\textbf{\textit{#1}}}}
\newcommand{\ConstantTok}[1]{\textcolor[rgb]{0.00,0.00,0.00}{#1}}
\newcommand{\ControlFlowTok}[1]{\textcolor[rgb]{0.13,0.29,0.53}{\textbf{#1}}}
\newcommand{\DataTypeTok}[1]{\textcolor[rgb]{0.13,0.29,0.53}{#1}}
\newcommand{\DecValTok}[1]{\textcolor[rgb]{0.00,0.00,0.81}{#1}}
\newcommand{\DocumentationTok}[1]{\textcolor[rgb]{0.56,0.35,0.01}{\textbf{\textit{#1}}}}
\newcommand{\ErrorTok}[1]{\textcolor[rgb]{0.64,0.00,0.00}{\textbf{#1}}}
\newcommand{\ExtensionTok}[1]{#1}
\newcommand{\FloatTok}[1]{\textcolor[rgb]{0.00,0.00,0.81}{#1}}
\newcommand{\FunctionTok}[1]{\textcolor[rgb]{0.00,0.00,0.00}{#1}}
\newcommand{\ImportTok}[1]{#1}
\newcommand{\InformationTok}[1]{\textcolor[rgb]{0.56,0.35,0.01}{\textbf{\textit{#1}}}}
\newcommand{\KeywordTok}[1]{\textcolor[rgb]{0.13,0.29,0.53}{\textbf{#1}}}
\newcommand{\NormalTok}[1]{#1}
\newcommand{\OperatorTok}[1]{\textcolor[rgb]{0.81,0.36,0.00}{\textbf{#1}}}
\newcommand{\OtherTok}[1]{\textcolor[rgb]{0.56,0.35,0.01}{#1}}
\newcommand{\PreprocessorTok}[1]{\textcolor[rgb]{0.56,0.35,0.01}{\textit{#1}}}
\newcommand{\RegionMarkerTok}[1]{#1}
\newcommand{\SpecialCharTok}[1]{\textcolor[rgb]{0.00,0.00,0.00}{#1}}
\newcommand{\SpecialStringTok}[1]{\textcolor[rgb]{0.31,0.60,0.02}{#1}}
\newcommand{\StringTok}[1]{\textcolor[rgb]{0.31,0.60,0.02}{#1}}
\newcommand{\VariableTok}[1]{\textcolor[rgb]{0.00,0.00,0.00}{#1}}
\newcommand{\VerbatimStringTok}[1]{\textcolor[rgb]{0.31,0.60,0.02}{#1}}
\newcommand{\WarningTok}[1]{\textcolor[rgb]{0.56,0.35,0.01}{\textbf{\textit{#1}}}}
\usepackage{longtable,booktabs}
% Correct order of tables after \paragraph or \subparagraph
\usepackage{etoolbox}
\makeatletter
\patchcmd\longtable{\par}{\if@noskipsec\mbox{}\fi\par}{}{}
\makeatother
% Allow footnotes in longtable head/foot
\IfFileExists{footnotehyper.sty}{\usepackage{footnotehyper}}{\usepackage{footnote}}
\makesavenoteenv{longtable}
\setlength{\emergencystretch}{3em} % prevent overfull lines
\providecommand{\tightlist}{%
  \setlength{\itemsep}{0pt}\setlength{\parskip}{0pt}}
\setcounter{secnumdepth}{5}

\usepackage{threeparttable}
\usepackage{threeparttablex}
\ifluatex
  \usepackage{selnolig}  % disable illegal ligatures
\fi
\usepackage[]{natbib}
\bibliographystyle{apalike}

\title{kableExtra 在 bookdown 项目中的使用}
\author{黄湘云}
\date{2020-09-29}

\begin{document}
\maketitle

{
\hypersetup{linkcolor=}
\setcounter{tocdepth}{1}
\tableofcontents
}
\hypertarget{table}{%
\chapter{表格}\label{table}}

这是一个稍微复杂一点的经典表格样式,常出现在论文或期刊的数值模拟比较部分\footnote{\url{https://arxiv.org/abs/1711.00437}}

\begin{Shaded}
\begin{Highlighting}[]
\KeywordTok{library}\NormalTok{(kableExtra)}
\NormalTok{db \textless{}{-}}\StringTok{ }\NormalTok{mtcars[, }\DecValTok{1}\OperatorTok{:}\DecValTok{7}\NormalTok{] }\OperatorTok{\%\textgreater{}\%}\StringTok{ }
\StringTok{  }\KeywordTok{transform}\NormalTok{(}\DataTypeTok{Methods =} \KeywordTok{rownames}\NormalTok{(.)) }\OperatorTok{\%\textgreater{}\%}\StringTok{ }
\StringTok{  \textasciigrave{}}\DataTypeTok{colnames\textless{}{-}}\StringTok{\textasciigrave{}}\NormalTok{(., }\DataTypeTok{value =} \KeywordTok{c}\NormalTok{(}\KeywordTok{rep}\NormalTok{(}\KeywordTok{c}\NormalTok{(}\StringTok{"Bias"}\NormalTok{, }\StringTok{"RMSE"}\NormalTok{), }\DecValTok{3}\NormalTok{), }\StringTok{""}\NormalTok{, }\StringTok{"Methods"}\NormalTok{))}
\end{Highlighting}
\end{Shaded}

\begin{Shaded}
\begin{Highlighting}[]
\KeywordTok{kable}\NormalTok{(db,}
  \DataTypeTok{format =} \StringTok{"latex"}\NormalTok{, }\DataTypeTok{booktabs =} \OtherTok{TRUE}\NormalTok{, }\DataTypeTok{escape =}\NormalTok{ T, }\DataTypeTok{row.names =}\NormalTok{ F,}
  \DataTypeTok{longtable =}\NormalTok{ T, }\DataTypeTok{caption =} \StringTok{"第1种类型的统计表格样式"}\NormalTok{,}
  \DataTypeTok{linesep =} \KeywordTok{c}\NormalTok{(}\StringTok{""}\NormalTok{, }\StringTok{""}\NormalTok{, }\StringTok{""}\NormalTok{, }\StringTok{""}\NormalTok{, }\StringTok{""}\NormalTok{, }\StringTok{"}\CharTok{\textbackslash{}\textbackslash{}}\StringTok{midrule"}\NormalTok{)}
\NormalTok{) }\OperatorTok{\%\textgreater{}\%}
\StringTok{  }\KeywordTok{kable\_styling}\NormalTok{(}
    \DataTypeTok{latex\_options =} \KeywordTok{c}\NormalTok{(}\StringTok{"hold\_position"}\NormalTok{, }\StringTok{"repeat\_header"}\NormalTok{),}
    \DataTypeTok{full\_width =}\NormalTok{ F, }\DataTypeTok{position =} \StringTok{"center"}\NormalTok{, }\DataTypeTok{repeat\_header\_method =} \StringTok{"replace"}\NormalTok{,}
    \DataTypeTok{repeat\_header\_text =} \StringTok{"续表@ref(tab:kableExtra{-}latex)"}
\NormalTok{  ) }\OperatorTok{\%\textgreater{}\%}
\StringTok{  }\KeywordTok{add\_header\_above}\NormalTok{(}\KeywordTok{c}\NormalTok{(}
    \StringTok{"$}\CharTok{\textbackslash{}\textbackslash{}\textbackslash{}\textbackslash{}}\StringTok{sigma\^{}2$"}\NormalTok{ =}\StringTok{ }\DecValTok{2}\NormalTok{, }\StringTok{"$}\CharTok{\textbackslash{}\textbackslash{}\textbackslash{}\textbackslash{}}\StringTok{phi$"}\NormalTok{ =}\StringTok{ }\DecValTok{2}\NormalTok{,}
    \StringTok{"$}\CharTok{\textbackslash{}\textbackslash{}\textbackslash{}\textbackslash{}}\StringTok{tau\^{}2$"}\NormalTok{ =}\StringTok{ }\DecValTok{2}\NormalTok{, }\StringTok{"$r=}\CharTok{\textbackslash{}\textbackslash{}\textbackslash{}\textbackslash{}}\StringTok{delta/}\CharTok{\textbackslash{}\textbackslash{}\textbackslash{}\textbackslash{}}\StringTok{phi$"}\NormalTok{ =}\StringTok{ }\DecValTok{1}\NormalTok{, }\StringTok{" "}
\NormalTok{  ), }\DataTypeTok{escape =}\NormalTok{ F) }\OperatorTok{\%\textgreater{}\%}
\StringTok{  }\KeywordTok{footnote}\NormalTok{(}
    \DataTypeTok{general\_title =} \StringTok{"注:"}\NormalTok{, }\DataTypeTok{title\_format =} \StringTok{"italic"}\NormalTok{, }\DataTypeTok{threeparttable =}\NormalTok{ T,}
    \DataTypeTok{general =} \StringTok{"* 星号表示的内容可以很长"}
\NormalTok{  )}
\end{Highlighting}
\end{Shaded}

\begin{ThreePartTable}
\begin{TableNotes}
\item \textit{注:} 
\item * 星号表示的内容可以很长
\end{TableNotes}
\begin{longtable}[t]{rrrrrrrl}
\caption{\label{tab:kableExtra-latex}第1种类型的统计表格样式}\\
\toprule
\multicolumn{2}{c}{$\sigma^2$} & \multicolumn{2}{c}{$\phi$} & \multicolumn{2}{c}{$\tau^2$} & \multicolumn{1}{c}{$r=\delta/\phi$} & \multicolumn{1}{c}{ } \\
\cmidrule(l{3pt}r{3pt}){1-2} \cmidrule(l{3pt}r{3pt}){3-4} \cmidrule(l{3pt}r{3pt}){5-6} \cmidrule(l{3pt}r{3pt}){7-7}
Bias & RMSE & Bias & RMSE & Bias & RMSE &  & Methods\\
\midrule
\endfirsthead
\caption[]{续表\ref{tab:kableExtra-latex}}\\
\toprule
Bias & RMSE & Bias & RMSE & Bias & RMSE &  & Methods\\
\midrule
\endhead

\endfoot
\bottomrule
\insertTableNotes
\endlastfoot
21.0 & 6 & 160.0 & 110 & 3.90 & 2.620 & 16.46 & Mazda RX4\\
21.0 & 6 & 160.0 & 110 & 3.90 & 2.875 & 17.02 & Mazda RX4 Wag\\
22.8 & 4 & 108.0 & 93 & 3.85 & 2.320 & 18.61 & Datsun 710\\
21.4 & 6 & 258.0 & 110 & 3.08 & 3.215 & 19.44 & Hornet 4 Drive\\
18.7 & 8 & 360.0 & 175 & 3.15 & 3.440 & 17.02 & Hornet Sportabout\\
18.1 & 6 & 225.0 & 105 & 2.76 & 3.460 & 20.22 & Valiant\\
\midrule
14.3 & 8 & 360.0 & 245 & 3.21 & 3.570 & 15.84 & Duster 360\\
24.4 & 4 & 146.7 & 62 & 3.69 & 3.190 & 20.00 & Merc 240D\\
22.8 & 4 & 140.8 & 95 & 3.92 & 3.150 & 22.90 & Merc 230\\
19.2 & 6 & 167.6 & 123 & 3.92 & 3.440 & 18.30 & Merc 280\\
17.8 & 6 & 167.6 & 123 & 3.92 & 3.440 & 18.90 & Merc 280C\\
16.4 & 8 & 275.8 & 180 & 3.07 & 4.070 & 17.40 & Merc 450SE\\
\midrule
17.3 & 8 & 275.8 & 180 & 3.07 & 3.730 & 17.60 & Merc 450SL\\
15.2 & 8 & 275.8 & 180 & 3.07 & 3.780 & 18.00 & Merc 450SLC\\
10.4 & 8 & 472.0 & 205 & 2.93 & 5.250 & 17.98 & Cadillac Fleetwood\\
10.4 & 8 & 460.0 & 215 & 3.00 & 5.424 & 17.82 & Lincoln Continental\\
14.7 & 8 & 440.0 & 230 & 3.23 & 5.345 & 17.42 & Chrysler Imperial\\
32.4 & 4 & 78.7 & 66 & 4.08 & 2.200 & 19.47 & Fiat 128\\
\midrule
30.4 & 4 & 75.7 & 52 & 4.93 & 1.615 & 18.52 & Honda Civic\\
33.9 & 4 & 71.1 & 65 & 4.22 & 1.835 & 19.90 & Toyota Corolla\\
21.5 & 4 & 120.1 & 97 & 3.70 & 2.465 & 20.01 & Toyota Corona\\
15.5 & 8 & 318.0 & 150 & 2.76 & 3.520 & 16.87 & Dodge Challenger\\
15.2 & 8 & 304.0 & 150 & 3.15 & 3.435 & 17.30 & AMC Javelin\\
13.3 & 8 & 350.0 & 245 & 3.73 & 3.840 & 15.41 & Camaro Z28\\
\midrule
19.2 & 8 & 400.0 & 175 & 3.08 & 3.845 & 17.05 & Pontiac Firebird\\
27.3 & 4 & 79.0 & 66 & 4.08 & 1.935 & 18.90 & Fiat X1-9\\
26.0 & 4 & 120.3 & 91 & 4.43 & 2.140 & 16.70 & Porsche 914-2\\
30.4 & 4 & 95.1 & 113 & 3.77 & 1.513 & 16.90 & Lotus Europa\\
15.8 & 8 & 351.0 & 264 & 4.22 & 3.170 & 14.50 & Ford Pantera L\\
19.7 & 6 & 145.0 & 175 & 3.62 & 2.770 & 15.50 & Ferrari Dino\\
\midrule
15.0 & 8 & 301.0 & 335 & 3.54 & 3.570 & 14.60 & Maserati Bora\\
21.4 & 4 & 121.0 & 109 & 4.11 & 2.780 & 18.60 & Volvo 142E\\*
\end{longtable}
\end{ThreePartTable}

\begin{itemize}
\tightlist
\item
  \texttt{striped} 默认使用浅灰色,\texttt{stripe\_color} 可以用来指定颜色 \texttt{stripe\_color="white"},它只在 LaTeX 下工作,HTML 下更改颜色需要设置 CSS,可以不使用 \texttt{striped} 改变默认的白底设置\\
\item
  \texttt{threeparttable\ =\ TRUE} 处理超长的注解标记,
\item
  \texttt{add\_header\_above} 函数内的 \texttt{escape\ =\ F} 用来处理数学公式,
\item
  \texttt{longtable\ =\ T} 表格很长时需要分页,因此使用续表,
\item
  \texttt{hold\_position} 使用了 \texttt{{[}!h{]}} 控制浮动
\item
  对于数学符号前要四个反斜杠这一点,作者今后可能会改变,只需要两个反斜杠,与 HTML 格式表格保持一致 \url{https://github.com/haozhu233/kableExtra/issues/120}
\item
  对某些数据用不同颜色高亮
  Selecting and colouring single table cells with \textbf{kableExtra} in R markdown \texttt{cell\_spec} \url{https://stackoverflow.com/questions/50118394}
\end{itemize}

\hypertarget{pandoc-table}{%
\section{Pandoc 支持的表格}\label{pandoc-table}}

其实 Pandoc's Markdown 本身也支持不少表格样式\footnote{\url{https://pandoc.org/MANUAL.html\#tables}},比如常见的简单表格样式,如表 \ref{tab:simple-tab} 和表 \ref{tab:another-tab}

\begin{longtable}[]{@{}rlcl@{}}
\caption{\label{tab:simple-tab} 简单表格语法展示}\tabularnewline
\toprule
Right & Left & Center & Default\tabularnewline
\midrule
\endfirsthead
\toprule
Right & Left & Center & Default\tabularnewline
\midrule
\endhead
12 & 12 & 12 & 12\tabularnewline
123 & 123 & 123 & 123\tabularnewline
1 & 1 & 1 & 1\tabularnewline
\bottomrule
\end{longtable}

再来一个表格

\begin{longtable}[]{@{}ll@{}}
\caption[\label{tab:another-tab} 表格标题可以含有脚注]{\label{tab:another-tab} 表格标题可以含有脚注\footnote{附有一个简短的脚注}}\tabularnewline
\toprule
First Header & Second Header\tabularnewline
\midrule
\endfirsthead
\toprule
First Header & Second Header\tabularnewline
\midrule
\endhead
Content Cell & Content Cell\tabularnewline
Content Cell & Content Cell\tabularnewline
\bottomrule
\end{longtable}

\hypertarget{session-info}{%
\section{软件信息}\label{session-info}}

\begin{Shaded}
\begin{Highlighting}[]
\KeywordTok{sessionInfo}\NormalTok{(}\KeywordTok{.packages}\NormalTok{(T))}
\end{Highlighting}
\end{Shaded}

\begin{verbatim}
## R version 4.0.2 (2020-06-22)
## Platform: x86_64-pc-linux-gnu (64-bit)
## Running under: Ubuntu 16.04.6 LTS
## 
## Matrix products: default
## BLAS:   /usr/lib/openblas-base/libblas.so.3
## LAPACK: /usr/lib/libopenblasp-r0.2.18.so
## 
## locale:
##  [1] LC_CTYPE=en_US.UTF-8       LC_NUMERIC=C              
##  [3] LC_TIME=en_US.UTF-8        LC_COLLATE=en_US.UTF-8    
##  [5] LC_MONETARY=en_US.UTF-8    LC_MESSAGES=en_US.UTF-8   
##  [7] LC_PAPER=en_US.UTF-8       LC_NAME=C                 
##  [9] LC_ADDRESS=C               LC_TELEPHONE=C            
## [11] LC_MEASUREMENT=en_US.UTF-8 LC_IDENTIFICATION=C       
## 
## attached base packages:
##  [1] base      compiler  datasets  graphics  grDevices grid      methods  
##  [8] parallel  splines   stats     stats4    tcltk     tools     utils    
## 
## other attached packages:
##  [1] askpass_1.1        base64enc_0.1-3    bookdown_0.20      callr_3.4.4       
##  [5] colorspace_1.4-1   curl_4.3           digest_0.6.25      evaluate_0.14     
##  [9] farver_2.0.3       glue_1.4.2         highr_0.8          htmltools_0.5.0   
## [13] httr_1.4.2         jsonlite_1.7.1     kableExtra_1.2.1   knitr_1.30        
## [17] labeling_0.3       lifecycle_0.2.0    magrittr_1.5       markdown_1.1      
## [21] mime_0.9           munsell_0.5.0      openssl_1.4.3      processx_3.4.4    
## [25] ps_1.3.4           R6_2.4.1           RColorBrewer_1.1-2 remotes_2.2.0     
## [29] rlang_0.4.7        rmarkdown_2.3      rstudioapi_0.11    rvest_0.3.6       
## [33] scales_1.1.1       selectr_0.4-2      stringi_1.5.3      stringr_1.4.0     
## [37] sys_3.4            tinytex_0.26       viridisLite_0.3.0  webshot_0.5.2     
## [41] xfun_0.17          xml2_1.3.2         yaml_2.2.1         boot_1.3-25       
## [45] class_7.3-17       cluster_2.1.0      codetools_0.2-16   foreign_0.8-80    
## [49] KernSmooth_2.23-17 lattice_0.20-41    MASS_7.3-51.6      Matrix_1.2-18     
## [53] mgcv_1.8-31        nlme_3.1-148       nnet_7.3-14        rpart_4.1-15      
## [57] spatial_7.3-12     survival_3.1-12
\end{verbatim}

  \bibliography{book.bib,packages.bib}

\end{document}
