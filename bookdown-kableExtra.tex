\documentclass[]{book}
\usepackage{lmodern}
\usepackage{amssymb,amsmath}
\usepackage{ifxetex,ifluatex}
\usepackage{fixltx2e} % provides \textsubscript
\ifnum 0\ifxetex 1\fi\ifluatex 1\fi=0 % if pdftex
  \usepackage[T1]{fontenc}
  \usepackage[utf8]{inputenc}
\else % if luatex or xelatex
  \ifxetex
    \usepackage{mathspec}
  \else
    \usepackage{fontspec}
  \fi
  \defaultfontfeatures{Ligatures=TeX,Scale=MatchLowercase}
\fi
% use upquote if available, for straight quotes in verbatim environments
\IfFileExists{upquote.sty}{\usepackage{upquote}}{}
% use microtype if available
\IfFileExists{microtype.sty}{%
\usepackage{microtype}
\UseMicrotypeSet[protrusion]{basicmath} % disable protrusion for tt fonts
}{}
\usepackage[margin=1in]{geometry}
\usepackage{hyperref}
\PassOptionsToPackage{usenames,dvipsnames}{color} % color is loaded by hyperref
\hypersetup{unicode=true,
            pdftitle={kableExtra 在 bookdown 项目中的使用},
            pdfauthor={黄湘云},
            colorlinks=true,
            linkcolor=Maroon,
            citecolor=Blue,
            urlcolor=Blue,
            breaklinks=true}
\urlstyle{same}  % don't use monospace font for urls
\usepackage{natbib}
\bibliographystyle{apalike}
\usepackage{color}
\usepackage{fancyvrb}
\newcommand{\VerbBar}{|}
\newcommand{\VERB}{\Verb[commandchars=\\\{\}]}
\DefineVerbatimEnvironment{Highlighting}{Verbatim}{commandchars=\\\{\}}
% Add ',fontsize=\small' for more characters per line
\usepackage{framed}
\definecolor{shadecolor}{RGB}{248,248,248}
\newenvironment{Shaded}{\begin{snugshade}}{\end{snugshade}}
\newcommand{\AlertTok}[1]{\textcolor[rgb]{0.94,0.16,0.16}{#1}}
\newcommand{\AnnotationTok}[1]{\textcolor[rgb]{0.56,0.35,0.01}{\textbf{\textit{#1}}}}
\newcommand{\AttributeTok}[1]{\textcolor[rgb]{0.77,0.63,0.00}{#1}}
\newcommand{\BaseNTok}[1]{\textcolor[rgb]{0.00,0.00,0.81}{#1}}
\newcommand{\BuiltInTok}[1]{#1}
\newcommand{\CharTok}[1]{\textcolor[rgb]{0.31,0.60,0.02}{#1}}
\newcommand{\CommentTok}[1]{\textcolor[rgb]{0.56,0.35,0.01}{\textit{#1}}}
\newcommand{\CommentVarTok}[1]{\textcolor[rgb]{0.56,0.35,0.01}{\textbf{\textit{#1}}}}
\newcommand{\ConstantTok}[1]{\textcolor[rgb]{0.00,0.00,0.00}{#1}}
\newcommand{\ControlFlowTok}[1]{\textcolor[rgb]{0.13,0.29,0.53}{\textbf{#1}}}
\newcommand{\DataTypeTok}[1]{\textcolor[rgb]{0.13,0.29,0.53}{#1}}
\newcommand{\DecValTok}[1]{\textcolor[rgb]{0.00,0.00,0.81}{#1}}
\newcommand{\DocumentationTok}[1]{\textcolor[rgb]{0.56,0.35,0.01}{\textbf{\textit{#1}}}}
\newcommand{\ErrorTok}[1]{\textcolor[rgb]{0.64,0.00,0.00}{\textbf{#1}}}
\newcommand{\ExtensionTok}[1]{#1}
\newcommand{\FloatTok}[1]{\textcolor[rgb]{0.00,0.00,0.81}{#1}}
\newcommand{\FunctionTok}[1]{\textcolor[rgb]{0.00,0.00,0.00}{#1}}
\newcommand{\ImportTok}[1]{#1}
\newcommand{\InformationTok}[1]{\textcolor[rgb]{0.56,0.35,0.01}{\textbf{\textit{#1}}}}
\newcommand{\KeywordTok}[1]{\textcolor[rgb]{0.13,0.29,0.53}{\textbf{#1}}}
\newcommand{\NormalTok}[1]{#1}
\newcommand{\OperatorTok}[1]{\textcolor[rgb]{0.81,0.36,0.00}{\textbf{#1}}}
\newcommand{\OtherTok}[1]{\textcolor[rgb]{0.56,0.35,0.01}{#1}}
\newcommand{\PreprocessorTok}[1]{\textcolor[rgb]{0.56,0.35,0.01}{\textit{#1}}}
\newcommand{\RegionMarkerTok}[1]{#1}
\newcommand{\SpecialCharTok}[1]{\textcolor[rgb]{0.00,0.00,0.00}{#1}}
\newcommand{\SpecialStringTok}[1]{\textcolor[rgb]{0.31,0.60,0.02}{#1}}
\newcommand{\StringTok}[1]{\textcolor[rgb]{0.31,0.60,0.02}{#1}}
\newcommand{\VariableTok}[1]{\textcolor[rgb]{0.00,0.00,0.00}{#1}}
\newcommand{\VerbatimStringTok}[1]{\textcolor[rgb]{0.31,0.60,0.02}{#1}}
\newcommand{\WarningTok}[1]{\textcolor[rgb]{0.56,0.35,0.01}{\textbf{\textit{#1}}}}
\usepackage{longtable,booktabs}
\usepackage{graphicx,grffile}
\makeatletter
\def\maxwidth{\ifdim\Gin@nat@width>\linewidth\linewidth\else\Gin@nat@width\fi}
\def\maxheight{\ifdim\Gin@nat@height>\textheight\textheight\else\Gin@nat@height\fi}
\makeatother
% Scale images if necessary, so that they will not overflow the page
% margins by default, and it is still possible to overwrite the defaults
% using explicit options in \includegraphics[width, height, ...]{}
\setkeys{Gin}{width=\maxwidth,height=\maxheight,keepaspectratio}
\IfFileExists{parskip.sty}{%
\usepackage{parskip}
}{% else
\setlength{\parindent}{0pt}
\setlength{\parskip}{6pt plus 2pt minus 1pt}
}
\setlength{\emergencystretch}{3em}  % prevent overfull lines
\providecommand{\tightlist}{%
  \setlength{\itemsep}{0pt}\setlength{\parskip}{0pt}}
\setcounter{secnumdepth}{5}
% Redefines (sub)paragraphs to behave more like sections
\ifx\paragraph\undefined\else
\let\oldparagraph\paragraph
\renewcommand{\paragraph}[1]{\oldparagraph{#1}\mbox{}}
\fi
\ifx\subparagraph\undefined\else
\let\oldsubparagraph\subparagraph
\renewcommand{\subparagraph}[1]{\oldsubparagraph{#1}\mbox{}}
\fi

%%% Use protect on footnotes to avoid problems with footnotes in titles
\let\rmarkdownfootnote\footnote%
\def\footnote{\protect\rmarkdownfootnote}

%%% Change title format to be more compact
\usepackage{titling}

% Create subtitle command for use in maketitle
\newcommand{\subtitle}[1]{
  \posttitle{
    \begin{center}\large#1\end{center}
    }
}

\setlength{\droptitle}{-2em}

  \title{kableExtra 在 bookdown 项目中的使用}
    \pretitle{\vspace{\droptitle}\centering\huge}
  \posttitle{\par}
    \author{黄湘云}
    \preauthor{\centering\large\emph}
  \postauthor{\par}
      \predate{\centering\large\emph}
  \postdate{\par}
    \date{2019-02-22}

\usepackage[UTF8, heading]{ctex}

\usepackage{array}
\usepackage{multirow}
\usepackage[table]{xcolor}
\usepackage{wrapfig}
\usepackage{float}
\usepackage{colortbl}
\usepackage{pdflscape}
\usepackage{tabu}
\usepackage{threeparttable}
\usepackage{threeparttablex}
\usepackage[normalem]{ulem}
\usepackage{makecell}

\begin{document}
\maketitle

{
\hypersetup{linkcolor=black}
\setcounter{tocdepth}{1}
\tableofcontents
}
\hypertarget{table}{%
\chapter{表格}\label{table}}

这是一个稍微复杂一点的经典表格样式,常出现在论文或期刊的数值模拟比较部分\footnote{\url{https://arxiv.org/abs/1711.00437}}

\begin{Shaded}
\begin{Highlighting}[]
\KeywordTok{library}\NormalTok{(dplyr)}
\KeywordTok{library}\NormalTok{(kableExtra)}
\NormalTok{db <-}\StringTok{ }\NormalTok{mtcars[, }\DecValTok{1}\OperatorTok{:}\DecValTok{7}\NormalTok{]}
\NormalTok{db2 <-}\StringTok{ }\KeywordTok{cbind}\NormalTok{(}\KeywordTok{rownames}\NormalTok{(db), db)}
\KeywordTok{colnames}\NormalTok{(db2) <-}\StringTok{ }\KeywordTok{c}\NormalTok{(}\StringTok{"Methods"}\NormalTok{, }\KeywordTok{rep}\NormalTok{(}\KeywordTok{c}\NormalTok{(}\StringTok{"Bias"}\NormalTok{, }\StringTok{"RMSE"}\NormalTok{), }\DecValTok{3}\NormalTok{), }\StringTok{""}\NormalTok{)}
\ControlFlowTok{if}\NormalTok{ (knitr}\OperatorTok{::}\KeywordTok{is_latex_output}\NormalTok{()) \{}
  \KeywordTok{kable}\NormalTok{(db2,}
    \DataTypeTok{format =} \StringTok{"latex"}\NormalTok{, }\DataTypeTok{booktabs =} \OtherTok{TRUE}\NormalTok{, }\DataTypeTok{escape =}\NormalTok{ T, }\DataTypeTok{row.names =}\NormalTok{ F,}
    \DataTypeTok{longtable =}\NormalTok{ T, }\DataTypeTok{caption =} \StringTok{"第1种类型的统计表格样式"}\NormalTok{,}
    \DataTypeTok{linesep =} \KeywordTok{c}\NormalTok{(}\StringTok{""}\NormalTok{, }\StringTok{""}\NormalTok{, }\StringTok{""}\NormalTok{, }\StringTok{""}\NormalTok{, }\StringTok{""}\NormalTok{, }\StringTok{"}\CharTok{\textbackslash{}\textbackslash{}}\StringTok{midrule"}\NormalTok{)}
\NormalTok{  ) }\OperatorTok
\StringTok{    }\KeywordTok{kable_styling}\NormalTok{(}
      \DataTypeTok{latex_options =} \KeywordTok{c}\NormalTok{(}\StringTok{"hold_position"}\NormalTok{, }\StringTok{"repeat_header"}\NormalTok{),}
      \DataTypeTok{full_width =}\NormalTok{ F, }\DataTypeTok{position =} \StringTok{"center"}\NormalTok{, }\DataTypeTok{repeat_header_method =} \StringTok{"replace"}\NormalTok{,}
      \DataTypeTok{repeat_header_text =} \StringTok{"续表\ref{tab:kableExtra}"}
\NormalTok{    ) }\OperatorTok
\StringTok{    }\KeywordTok{add_header_above}\NormalTok{(}\KeywordTok{c}\NormalTok{(}\StringTok{" "}\NormalTok{,}
      \StringTok{"$}\CharTok{\textbackslash{}\textbackslash{}\textbackslash{}\textbackslash{}}\StringTok{sigma^2$"}\NormalTok{ =}\StringTok{ }\DecValTok{2}\NormalTok{, }\StringTok{"$}\CharTok{\textbackslash{}\textbackslash{}\textbackslash{}\textbackslash{}}\StringTok{phi$"}\NormalTok{ =}\StringTok{ }\DecValTok{2}\NormalTok{,}
      \StringTok{"$}\CharTok{\textbackslash{}\textbackslash{}\textbackslash{}\textbackslash{}}\StringTok{tau^2$"}\NormalTok{ =}\StringTok{ }\DecValTok{2}\NormalTok{, }\StringTok{"$r=}\CharTok{\textbackslash{}\textbackslash{}\textbackslash{}\textbackslash{}}\StringTok{delta/}\CharTok{\textbackslash{}\textbackslash{}\textbackslash{}\textbackslash{}}\StringTok{phi$"}\NormalTok{ =}\StringTok{ }\DecValTok{1}
\NormalTok{    ), }\DataTypeTok{escape =}\NormalTok{ F) }\OperatorTok
\StringTok{    }\KeywordTok{footnote}\NormalTok{(}
      \DataTypeTok{general_title =} \StringTok{"注:"}\NormalTok{, }\DataTypeTok{title_format =} \StringTok{"italic"}\NormalTok{, }\DataTypeTok{threeparttable =}\NormalTok{ T,}
      \DataTypeTok{general =} \StringTok{"* 星号表示的内容可以很长"}
\NormalTok{    )}
\NormalTok{\} }\ControlFlowTok{else}\NormalTok{ \{}
  \KeywordTok{kable}\NormalTok{(db2,}
    \DataTypeTok{format =} \StringTok{"html"}\NormalTok{, }\DataTypeTok{booktabs =} \OtherTok{TRUE}\NormalTok{, }\DataTypeTok{escape =}\NormalTok{ T, }\DataTypeTok{row.names =}\NormalTok{ F,}
    \DataTypeTok{caption =} \StringTok{"第1种类型的统计表格样式"}
\NormalTok{  ) }\OperatorTok
\StringTok{    }\KeywordTok{kable_styling}\NormalTok{(}
      \DataTypeTok{bootstrap_options =} \KeywordTok{c}\NormalTok{(}\StringTok{"basic"}\NormalTok{),}
      \DataTypeTok{full_width =}\NormalTok{ F, }\DataTypeTok{position =} \StringTok{"center"}
\NormalTok{    ) }\OperatorTok
\StringTok{    }\KeywordTok{add_header_above}\NormalTok{(}\KeywordTok{c}\NormalTok{(}\StringTok{""}\NormalTok{,}
      \StringTok{"$}\CharTok{\textbackslash{}\textbackslash{}}\StringTok{sigma^2$"}\NormalTok{ =}\StringTok{ }\DecValTok{2}\NormalTok{, }\StringTok{"$}\CharTok{\textbackslash{}\textbackslash{}}\StringTok{phi$"}\NormalTok{ =}\StringTok{ }\DecValTok{2}\NormalTok{,}
      \StringTok{"$}\CharTok{\textbackslash{}\textbackslash{}}\StringTok{tau^2$"}\NormalTok{ =}\StringTok{ }\DecValTok{2}\NormalTok{, }\StringTok{"$r=}\CharTok{\textbackslash{}\textbackslash{}}\StringTok{delta/}\CharTok{\textbackslash{}\textbackslash{}}\StringTok{phi$"}\NormalTok{ =}\StringTok{ }\DecValTok{1}
\NormalTok{    ), }\DataTypeTok{escape =}\NormalTok{ F) }\OperatorTok
\StringTok{    }\KeywordTok{footnote}\NormalTok{(}
      \DataTypeTok{general_title =} \StringTok{"注:"}\NormalTok{, }\DataTypeTok{title_format =} \StringTok{"italic"}\NormalTok{, }\DataTypeTok{threeparttable =}\NormalTok{ T,}
      \DataTypeTok{general =} \StringTok{"* 星号表示的内容可以很长"}
\NormalTok{    )}
\NormalTok{\}}
\end{Highlighting}
\end{Shaded}

\begin{ThreePartTable}
\begin{TableNotes}
\item \textit{注:} 
\item * 星号表示的内容可以很长
\end{TableNotes}
\begin{longtable}{lrrrrrrr}
\caption{\label{tab:kableExtra}第1种类型的统计表格样式}\\
\toprule
\multicolumn{1}{c}{ } & \multicolumn{2}{c}{$\sigma^2$} & \multicolumn{2}{c}{$\phi$} & \multicolumn{2}{c}{$\tau^2$} & \multicolumn{1}{c}{$r=\delta/\phi$} \\
\cmidrule(l{3pt}r{3pt}){2-3} \cmidrule(l{3pt}r{3pt}){4-5} \cmidrule(l{3pt}r{3pt}){6-7} \cmidrule(l{3pt}r{3pt}){8-8}
Methods & Bias & RMSE & Bias & RMSE & Bias & RMSE & \\
\midrule
\endfirsthead
\caption[]{续表\ref{tab:kableExtra}}\\
\toprule
Methods & Bias & RMSE & Bias & RMSE & Bias & RMSE & \\
\midrule
\endhead
\
\endfoot
\bottomrule
\insertTableNotes
\endlastfoot
Mazda RX4 & 21.0 & 6 & 160.0 & 110 & 3.90 & 2.620 & 16.46\\
Mazda RX4 Wag & 21.0 & 6 & 160.0 & 110 & 3.90 & 2.875 & 17.02\\
Datsun 710 & 22.8 & 4 & 108.0 & 93 & 3.85 & 2.320 & 18.61\\
Hornet 4 Drive & 21.4 & 6 & 258.0 & 110 & 3.08 & 3.215 & 19.44\\
Hornet Sportabout & 18.7 & 8 & 360.0 & 175 & 3.15 & 3.440 & 17.02\\
Valiant & 18.1 & 6 & 225.0 & 105 & 2.76 & 3.460 & 20.22\\
\midrule
Duster 360 & 14.3 & 8 & 360.0 & 245 & 3.21 & 3.570 & 15.84\\
Merc 240D & 24.4 & 4 & 146.7 & 62 & 3.69 & 3.190 & 20.00\\
Merc 230 & 22.8 & 4 & 140.8 & 95 & 3.92 & 3.150 & 22.90\\
Merc 280 & 19.2 & 6 & 167.6 & 123 & 3.92 & 3.440 & 18.30\\
Merc 280C & 17.8 & 6 & 167.6 & 123 & 3.92 & 3.440 & 18.90\\
Merc 450SE & 16.4 & 8 & 275.8 & 180 & 3.07 & 4.070 & 17.40\\
\midrule
Merc 450SL & 17.3 & 8 & 275.8 & 180 & 3.07 & 3.730 & 17.60\\
Merc 450SLC & 15.2 & 8 & 275.8 & 180 & 3.07 & 3.780 & 18.00\\
Cadillac Fleetwood & 10.4 & 8 & 472.0 & 205 & 2.93 & 5.250 & 17.98\\
Lincoln Continental & 10.4 & 8 & 460.0 & 215 & 3.00 & 5.424 & 17.82\\
Chrysler Imperial & 14.7 & 8 & 440.0 & 230 & 3.23 & 5.345 & 17.42\\
Fiat 128 & 32.4 & 4 & 78.7 & 66 & 4.08 & 2.200 & 19.47\\
\midrule
Honda Civic & 30.4 & 4 & 75.7 & 52 & 4.93 & 1.615 & 18.52\\
Toyota Corolla & 33.9 & 4 & 71.1 & 65 & 4.22 & 1.835 & 19.90\\
Toyota Corona & 21.5 & 4 & 120.1 & 97 & 3.70 & 2.465 & 20.01\\
Dodge Challenger & 15.5 & 8 & 318.0 & 150 & 2.76 & 3.520 & 16.87\\
AMC Javelin & 15.2 & 8 & 304.0 & 150 & 3.15 & 3.435 & 17.30\\
Camaro Z28 & 13.3 & 8 & 350.0 & 245 & 3.73 & 3.840 & 15.41\\
\midrule
Pontiac Firebird & 19.2 & 8 & 400.0 & 175 & 3.08 & 3.845 & 17.05\\
Fiat X1-9 & 27.3 & 4 & 79.0 & 66 & 4.08 & 1.935 & 18.90\\
Porsche 914-2 & 26.0 & 4 & 120.3 & 91 & 4.43 & 2.140 & 16.70\\
Lotus Europa & 30.4 & 4 & 95.1 & 113 & 3.77 & 1.513 & 16.90\\
Ford Pantera L & 15.8 & 8 & 351.0 & 264 & 4.22 & 3.170 & 14.50\\
Ferrari Dino & 19.7 & 6 & 145.0 & 175 & 3.62 & 2.770 & 15.50\\
\midrule
Maserati Bora & 15.0 & 8 & 301.0 & 335 & 3.54 & 3.570 & 14.60\\
Volvo 142E & 21.4 & 4 & 121.0 & 109 & 4.11 & 2.780 & 18.60\\*
\end{longtable}
\end{ThreePartTable}

\begin{itemize}
\tightlist
\item
  \texttt{striped} 默认使用浅灰色,\texttt{stripe\_color} 可以用来指定颜色 \texttt{stripe\_color="white"},它只在 LaTeX 下工作,HTML 下更改颜色需要设置 CSS,可以不使用 \texttt{striped} 改变默认的白底设置\\
\item
  \texttt{threeparttable\ =\ TRUE} 处理超长的注解标记,
\item
  \texttt{add\_header\_above} 函数内的 \texttt{escape\ =\ F} 用来处理数学公式,
\item
  \texttt{longtable\ =\ T} 表格很长时需要分页,因此使用续表,
\item
  \texttt{hold\_position} 使用了 \texttt{{[}!h{]}} 控制浮动
\item
  对于数学符号前要四个反斜杠这一点,作者今后可能会改变,只需要两个反斜杠,与 HTML 格式表格保持一致 \url{https://github.com/haozhu233/kableExtra/issues/120}
\item
  对某些数据用不同颜色高亮
  Selecting and colouring single table cells with \textbf{kableExtra} in R markdown \texttt{cell\_spec} \url{https://stackoverflow.com/questions/50118394}
\end{itemize}

其实 Pandoc's Markdown 本身也支持不少表格样式\footnote{\url{https://pandoc.org/MANUAL.html\#tables}},少见的比如

\begin{longtable}[]{@{}clrl@{}}
\caption{没有编号的可以跨行的表格.}\tabularnewline
\toprule
\endhead
\begin{minipage}[t]{0.15\columnwidth}\centering
First\strut
\end{minipage} & \begin{minipage}[t]{0.10\columnwidth}\raggedright
row\strut
\end{minipage} & \begin{minipage}[t]{0.20\columnwidth}\raggedleft
12.0\strut
\end{minipage} & \begin{minipage}[t]{0.32\columnwidth}\raggedright
Example of a row that
spans multiple lines.\strut
\end{minipage}\tabularnewline
\begin{minipage}[t]{0.15\columnwidth}\centering
Second\strut
\end{minipage} & \begin{minipage}[t]{0.10\columnwidth}\raggedright
row\strut
\end{minipage} & \begin{minipage}[t]{0.20\columnwidth}\raggedleft
5.0\strut
\end{minipage} & \begin{minipage}[t]{0.32\columnwidth}\raggedright
Here's another one. Note
the blank line between
rows.\strut
\end{minipage}\tabularnewline
\bottomrule
\end{longtable}

\hypertarget{session-info}{%
\section{软件信息}\label{session-info}}

\begin{Shaded}
\begin{Highlighting}[]
\NormalTok{devtools}\OperatorTok{::}\KeywordTok{session_info}\NormalTok{(}\KeywordTok{.packages}\NormalTok{(T))}
\end{Highlighting}
\end{Shaded}

\begin{verbatim}
## - Session info ----------------------------------------------------------
##  setting  value                       
##  version  R version 3.5.2 (2017-01-27)
##  os       Ubuntu 14.04.5 LTS          
##  system   x86_64, linux-gnu           
##  ui       X11                         
##  language (EN)                        
##  collate  en_US.UTF-8                 
##  ctype    en_US.UTF-8                 
##  tz       UTC                         
##  date     2019-02-22                  
## 
## - Packages --------------------------------------------------------------
##  package      * version  date       lib source        
##  askpass        1.1      2019-01-13 [1] CRAN (R 3.5.2)
##  assertthat     0.2.0    2017-04-11 [1] CRAN (R 3.5.2)
##  backports      1.1.3    2018-12-14 [1] CRAN (R 3.5.2)
##  base64enc      0.1-3    2015-07-28 [1] CRAN (R 3.5.2)
##  BH             1.69.0-1 2019-01-07 [1] CRAN (R 3.5.2)
##  bookdown       0.9      2018-12-21 [1] CRAN (R 3.5.2)
##  boot           1.3-20   2017-08-06 [3] CRAN (R 3.5.2)
##  callr          3.1.1    2018-12-21 [1] CRAN (R 3.5.2)
##  class          7.3-14   2015-08-30 [3] CRAN (R 3.5.2)
##  cli            1.0.1    2018-09-25 [1] CRAN (R 3.5.2)
##  clipr          0.5.0    2019-01-11 [1] CRAN (R 3.5.2)
##  clisymbols     1.2.0    2017-05-21 [1] CRAN (R 3.5.2)
##  cluster        2.0.7-1  2018-04-13 [3] CRAN (R 3.5.2)
##  codetools      0.2-15   2016-10-05 [3] CRAN (R 3.5.2)
##  colorspace     1.4-0    2019-01-13 [1] CRAN (R 3.5.2)
##  crayon         1.3.4    2017-09-16 [1] CRAN (R 3.5.2)
##  curl           3.3      2019-01-10 [1] CRAN (R 3.5.2)
##  desc           1.2.0    2018-05-01 [1] CRAN (R 3.5.2)
##  devtools       2.0.1    2018-10-26 [1] CRAN (R 3.5.2)
##  digest         0.6.18   2018-10-10 [1] CRAN (R 3.5.2)
##  dplyr        * 0.8.0.1  2019-02-15 [1] CRAN (R 3.5.2)
##  evaluate       0.13     2019-02-12 [1] CRAN (R 3.5.2)
##  fansi          0.4.0    2018-10-05 [1] CRAN (R 3.5.2)
##  foreign        0.8-71   2018-07-20 [3] CRAN (R 3.5.2)
##  fs             1.2.6    2018-08-23 [1] CRAN (R 3.5.2)
##  gh             1.0.1    2017-07-16 [1] CRAN (R 3.5.2)
##  git2r          0.24.0   2019-01-07 [1] CRAN (R 3.5.2)
##  glue           1.3.0    2018-07-17 [1] CRAN (R 3.5.2)
##  highr          0.7      2018-06-09 [1] CRAN (R 3.5.2)
##  hms            0.4.2    2018-03-10 [1] CRAN (R 3.5.2)
##  htmltools      0.3.6    2017-04-28 [1] CRAN (R 3.5.2)
##  httr           1.4.0    2018-12-11 [1] CRAN (R 3.5.2)
##  ini            0.3.1    2018-05-20 [1] CRAN (R 3.5.2)
##  jsonlite       1.6      2018-12-07 [1] CRAN (R 3.5.2)
##  kableExtra   * 1.0.1    2019-01-22 [1] CRAN (R 3.5.2)
##  KernSmooth     2.23-15  2015-06-29 [3] CRAN (R 3.5.2)
##  knitr          1.21     2018-12-10 [1] CRAN (R 3.5.2)
##  labeling       0.3      2014-08-23 [1] CRAN (R 3.5.2)
##  lattice        0.20-38  2018-11-04 [3] CRAN (R 3.5.2)
##  magrittr       1.5      2014-11-22 [1] CRAN (R 3.5.2)
##  markdown       0.9      2018-12-07 [1] CRAN (R 3.5.2)
##  MASS           7.3-51.1 2018-11-01 [3] CRAN (R 3.5.2)
##  Matrix         1.2-15   2018-11-01 [3] CRAN (R 3.5.2)
##  memoise        1.1.0    2017-04-21 [1] CRAN (R 3.5.2)
##  mgcv           1.8-26   2018-11-21 [3] CRAN (R 3.5.2)
##  mime           0.6      2018-10-05 [1] CRAN (R 3.5.2)
##  munsell        0.5.0    2018-06-12 [1] CRAN (R 3.5.2)
##  nlme           3.1-137  2018-04-07 [3] CRAN (R 3.5.2)
##  nnet           7.3-12   2016-02-02 [3] CRAN (R 3.5.2)
##  openssl        1.2.1    2019-01-17 [1] CRAN (R 3.5.2)
##  pillar         1.3.1    2018-12-15 [1] CRAN (R 3.5.2)
##  pkgbuild       1.0.2    2018-10-16 [1] CRAN (R 3.5.2)
##  pkgconfig      2.0.2    2018-08-16 [1] CRAN (R 3.5.2)
##  pkgload        1.0.2    2018-10-29 [1] CRAN (R 3.5.2)
##  plogr          0.2.0    2018-03-25 [1] CRAN (R 3.5.2)
##  prettyunits    1.0.2    2015-07-13 [1] CRAN (R 3.5.2)
##  processx       3.2.1    2018-12-05 [1] CRAN (R 3.5.2)
##  ps             1.3.0    2018-12-21 [1] CRAN (R 3.5.2)
##  purrr          0.3.0    2019-01-27 [1] CRAN (R 3.5.2)
##  R6             2.4.0    2019-02-14 [1] CRAN (R 3.5.2)
##  rcmdcheck      1.3.2    2018-11-10 [1] CRAN (R 3.5.2)
##  RColorBrewer   1.1-2    2014-12-07 [1] CRAN (R 3.5.2)
##  Rcpp           1.0.0    2018-11-07 [1] CRAN (R 3.5.2)
##  readr          1.3.1    2018-12-21 [1] CRAN (R 3.5.2)
##  remotes        2.0.2    2018-10-30 [1] CRAN (R 3.5.2)
##  rlang          0.3.1    2019-01-08 [1] CRAN (R 3.5.2)
##  rmarkdown      1.11     2018-12-08 [1] CRAN (R 3.5.2)
##  rpart          4.1-13   2018-02-23 [3] CRAN (R 3.5.2)
##  rprojroot      1.3-2    2018-01-03 [1] CRAN (R 3.5.2)
##  rstudioapi     0.9.0    2019-01-09 [1] CRAN (R 3.5.2)
##  rvest          0.3.2    2016-06-17 [1] CRAN (R 3.5.2)
##  scales         1.0.0    2018-08-09 [1] CRAN (R 3.5.2)
##  selectr        0.4-1    2018-04-06 [1] CRAN (R 3.5.2)
##  sessioninfo    1.1.1    2018-11-05 [1] CRAN (R 3.5.2)
##  spatial        7.3-11   2015-08-30 [3] CRAN (R 3.5.2)
##  stringi        1.3.1    2019-02-13 [1] CRAN (R 3.5.2)
##  stringr        1.4.0    2019-02-10 [1] CRAN (R 3.5.2)
##  survival       2.43-3   2018-11-26 [3] CRAN (R 3.5.2)
##  sys            2.1      2018-11-13 [1] CRAN (R 3.5.2)
##  tibble         2.0.1    2019-01-12 [1] CRAN (R 3.5.2)
##  tidyselect     0.2.5    2018-10-11 [1] CRAN (R 3.5.2)
##  tinytex        0.10     2019-01-10 [1] CRAN (R 3.5.2)
##  usethis        1.4.0    2018-08-14 [1] CRAN (R 3.5.2)
##  utf8           1.1.4    2018-05-24 [1] CRAN (R 3.5.2)
##  viridisLite    0.3.0    2018-02-01 [1] CRAN (R 3.5.2)
##  webshot        0.5.1    2018-09-28 [1] CRAN (R 3.5.2)
##  whisker        0.3-2    2013-04-28 [1] CRAN (R 3.5.2)
##  withr          2.1.2    2018-03-15 [1] CRAN (R 3.5.2)
##  xfun           0.5      2019-02-20 [1] CRAN (R 3.5.2)
##  xml2           1.2.0    2018-01-24 [1] CRAN (R 3.5.2)
##  xopen          1.0.0    2018-09-17 [1] CRAN (R 3.5.2)
##  yaml           2.2.0    2018-07-25 [1] CRAN (R 3.5.2)
## 
## [1] /home/travis/R/Library
## [2] /usr/local/lib/R/site-library
## [3] /home/travis/R-bin/lib/R/library
\end{verbatim}

\bibliography{book.bib,packages.bib}


\end{document}
