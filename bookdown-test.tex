\documentclass[]{book}
\usepackage{lmodern}
\usepackage{amssymb,amsmath}
\usepackage{ifxetex,ifluatex}
\usepackage{fixltx2e} % provides \textsubscript
\ifnum 0\ifxetex 1\fi\ifluatex 1\fi=0 % if pdftex
  \usepackage[T1]{fontenc}
  \usepackage[utf8]{inputenc}
\else % if luatex or xelatex
  \ifxetex
    \usepackage{mathspec}
  \else
    \usepackage{fontspec}
  \fi
  \defaultfontfeatures{Ligatures=TeX,Scale=MatchLowercase}
\fi
% use upquote if available, for straight quotes in verbatim environments
\IfFileExists{upquote.sty}{\usepackage{upquote}}{}
% use microtype if available
\IfFileExists{microtype.sty}{%
\usepackage{microtype}
\UseMicrotypeSet[protrusion]{basicmath} % disable protrusion for tt fonts
}{}
\usepackage[margin=1in]{geometry}
\usepackage{hyperref}
\PassOptionsToPackage{usenames,dvipsnames}{color} % color is loaded by hyperref
\hypersetup{unicode=true,
            pdftitle={A Minimal Book Example},
            pdfauthor={Yihui Xie},
            colorlinks=true,
            linkcolor=Maroon,
            citecolor=Blue,
            urlcolor=Blue,
            breaklinks=true}
\urlstyle{same}  % don't use monospace font for urls
\usepackage{natbib}
\bibliographystyle{apalike}
\usepackage{color}
\usepackage{fancyvrb}
\newcommand{\VerbBar}{|}
\newcommand{\VERB}{\Verb[commandchars=\\\{\}]}
\DefineVerbatimEnvironment{Highlighting}{Verbatim}{commandchars=\\\{\}}
% Add ',fontsize=\small' for more characters per line
\usepackage{framed}
\definecolor{shadecolor}{RGB}{248,248,248}
\newenvironment{Shaded}{\begin{snugshade}}{\end{snugshade}}
\newcommand{\AlertTok}[1]{\textcolor[rgb]{0.94,0.16,0.16}{#1}}
\newcommand{\AnnotationTok}[1]{\textcolor[rgb]{0.56,0.35,0.01}{\textbf{\textit{#1}}}}
\newcommand{\AttributeTok}[1]{\textcolor[rgb]{0.77,0.63,0.00}{#1}}
\newcommand{\BaseNTok}[1]{\textcolor[rgb]{0.00,0.00,0.81}{#1}}
\newcommand{\BuiltInTok}[1]{#1}
\newcommand{\CharTok}[1]{\textcolor[rgb]{0.31,0.60,0.02}{#1}}
\newcommand{\CommentTok}[1]{\textcolor[rgb]{0.56,0.35,0.01}{\textit{#1}}}
\newcommand{\CommentVarTok}[1]{\textcolor[rgb]{0.56,0.35,0.01}{\textbf{\textit{#1}}}}
\newcommand{\ConstantTok}[1]{\textcolor[rgb]{0.00,0.00,0.00}{#1}}
\newcommand{\ControlFlowTok}[1]{\textcolor[rgb]{0.13,0.29,0.53}{\textbf{#1}}}
\newcommand{\DataTypeTok}[1]{\textcolor[rgb]{0.13,0.29,0.53}{#1}}
\newcommand{\DecValTok}[1]{\textcolor[rgb]{0.00,0.00,0.81}{#1}}
\newcommand{\DocumentationTok}[1]{\textcolor[rgb]{0.56,0.35,0.01}{\textbf{\textit{#1}}}}
\newcommand{\ErrorTok}[1]{\textcolor[rgb]{0.64,0.00,0.00}{\textbf{#1}}}
\newcommand{\ExtensionTok}[1]{#1}
\newcommand{\FloatTok}[1]{\textcolor[rgb]{0.00,0.00,0.81}{#1}}
\newcommand{\FunctionTok}[1]{\textcolor[rgb]{0.00,0.00,0.00}{#1}}
\newcommand{\ImportTok}[1]{#1}
\newcommand{\InformationTok}[1]{\textcolor[rgb]{0.56,0.35,0.01}{\textbf{\textit{#1}}}}
\newcommand{\KeywordTok}[1]{\textcolor[rgb]{0.13,0.29,0.53}{\textbf{#1}}}
\newcommand{\NormalTok}[1]{#1}
\newcommand{\OperatorTok}[1]{\textcolor[rgb]{0.81,0.36,0.00}{\textbf{#1}}}
\newcommand{\OtherTok}[1]{\textcolor[rgb]{0.56,0.35,0.01}{#1}}
\newcommand{\PreprocessorTok}[1]{\textcolor[rgb]{0.56,0.35,0.01}{\textit{#1}}}
\newcommand{\RegionMarkerTok}[1]{#1}
\newcommand{\SpecialCharTok}[1]{\textcolor[rgb]{0.00,0.00,0.00}{#1}}
\newcommand{\SpecialStringTok}[1]{\textcolor[rgb]{0.31,0.60,0.02}{#1}}
\newcommand{\StringTok}[1]{\textcolor[rgb]{0.31,0.60,0.02}{#1}}
\newcommand{\VariableTok}[1]{\textcolor[rgb]{0.00,0.00,0.00}{#1}}
\newcommand{\VerbatimStringTok}[1]{\textcolor[rgb]{0.31,0.60,0.02}{#1}}
\newcommand{\WarningTok}[1]{\textcolor[rgb]{0.56,0.35,0.01}{\textbf{\textit{#1}}}}
\usepackage{longtable,booktabs}
\usepackage{graphicx,grffile}
\makeatletter
\def\maxwidth{\ifdim\Gin@nat@width>\linewidth\linewidth\else\Gin@nat@width\fi}
\def\maxheight{\ifdim\Gin@nat@height>\textheight\textheight\else\Gin@nat@height\fi}
\makeatother
% Scale images if necessary, so that they will not overflow the page
% margins by default, and it is still possible to overwrite the defaults
% using explicit options in \includegraphics[width, height, ...]{}
\setkeys{Gin}{width=\maxwidth,height=\maxheight,keepaspectratio}
\IfFileExists{parskip.sty}{%
\usepackage{parskip}
}{% else
\setlength{\parindent}{0pt}
\setlength{\parskip}{6pt plus 2pt minus 1pt}
}
\setlength{\emergencystretch}{3em}  % prevent overfull lines
\providecommand{\tightlist}{%
  \setlength{\itemsep}{0pt}\setlength{\parskip}{0pt}}
\setcounter{secnumdepth}{5}
% Redefines (sub)paragraphs to behave more like sections
\ifx\paragraph\undefined\else
\let\oldparagraph\paragraph
\renewcommand{\paragraph}[1]{\oldparagraph{#1}\mbox{}}
\fi
\ifx\subparagraph\undefined\else
\let\oldsubparagraph\subparagraph
\renewcommand{\subparagraph}[1]{\oldsubparagraph{#1}\mbox{}}
\fi

%%% Use protect on footnotes to avoid problems with footnotes in titles
\let\rmarkdownfootnote\footnote%
\def\footnote{\protect\rmarkdownfootnote}

%%% Change title format to be more compact
\usepackage{titling}

% Create subtitle command for use in maketitle
\newcommand{\subtitle}[1]{
  \posttitle{
    \begin{center}\large#1\end{center}
    }
}

\setlength{\droptitle}{-2em}

  \title{A Minimal Book Example}
    \pretitle{\vspace{\droptitle}\centering\huge}
  \posttitle{\par}
    \author{Yihui Xie}
    \preauthor{\centering\large\emph}
  \postauthor{\par}
      \predate{\centering\large\emph}
  \postdate{\par}
    \date{2018-09-01}

\usepackage[UTF8, heading]{ctex}

\usepackage{array}
\usepackage{multirow}
\usepackage[table]{xcolor}
\usepackage{wrapfig}
\usepackage{float}
\usepackage{colortbl}
\usepackage{pdflscape}
\usepackage{tabu}
\usepackage{threeparttable}
\usepackage{threeparttablex}
\usepackage[normalem]{ulem}
\usepackage{makecell}

\begin{document}
\maketitle

{
\hypersetup{linkcolor=black}
\setcounter{tocdepth}{1}
\tableofcontents
}
\hypertarget{prerequisites}{%
\chapter{Prerequisites}\label{prerequisites}}

This is a \emph{sample} book written in \textbf{Markdown}. You can use
anything that Pandoc's Markdown supports, e.g., a math equation
\(a^2 + b^2 = c^2\).

\[
\begin{array}{rcll}
p(\boldsymbol{\theta}|Y)  & =  & \displaystyle \frac{p(\boldsymbol{\theta},Y)}{p(Y)}
& \mbox{ [条件概率定义]}
\\[4pt]
& = & \displaystyle \frac{p(Y|\boldsymbol{\theta}) p(\boldsymbol{\theta})}{p(Y)}
& \mbox{ [链式法则]}
\\[4pt]
& = & \displaystyle \frac{p(Y|\boldsymbol{\theta})p(\boldsymbol{\theta})}{\int_{\Theta}p(Y,\boldsymbol{\theta})d\boldsymbol{\theta}}
& \mbox{ [全概率公式]}
\\[4pt]
& = & \displaystyle \frac{p(Y|\boldsymbol{\theta})p(\boldsymbol{\theta})}{\int_{\Theta}p(Y|\boldsymbol{\theta})p(\boldsymbol{\theta})d\boldsymbol{\theta}}
& \mbox{ [链式法则]}
\\[4pt]
& \propto & \displaystyle p(Y|\boldsymbol{\theta})p(\boldsymbol{\theta})
& \mbox{ [Y 是固定的]}
\end{array}
\]

The \textbf{bookdown} package can be installed from CRAN or Github:

\begin{Shaded}
\begin{Highlighting}[]
\KeywordTok{install.packages}\NormalTok{(}\StringTok{"bookdown"}\NormalTok{)}
\CommentTok{# or the development version}
\CommentTok{# devtools::install_github("rstudio/bookdown")}
\end{Highlighting}
\end{Shaded}

插入横线,续表的中文化

\begin{Shaded}
\begin{Highlighting}[]
\KeywordTok{library}\NormalTok{(kableExtra)}
\NormalTok{db <-}\StringTok{ }\NormalTok{mtcars[, }\DecValTok{1}\OperatorTok{:}\DecValTok{7}\NormalTok{]}
\NormalTok{db2 <-}\StringTok{ }\KeywordTok{cbind}\NormalTok{(}\KeywordTok{rownames}\NormalTok{(db), db)}
\KeywordTok{colnames}\NormalTok{(db2) <-}\StringTok{ }\KeywordTok{c}\NormalTok{(}\StringTok{"Methods"}\NormalTok{, }\KeywordTok{rep}\NormalTok{(}\KeywordTok{c}\NormalTok{(}\StringTok{"Bias"}\NormalTok{, }\StringTok{"RMSE"}\NormalTok{), }\DecValTok{3}\NormalTok{), }\StringTok{""}\NormalTok{)}

\KeywordTok{kable}\NormalTok{(db2,}
  \DataTypeTok{format =} \StringTok{"latex"}\NormalTok{, }\DataTypeTok{booktabs =} \OtherTok{TRUE}\NormalTok{, }\DataTypeTok{escape =}\NormalTok{ T, }\DataTypeTok{row.names =}\NormalTok{ F,}
  \DataTypeTok{longtable =}\NormalTok{ T, }\DataTypeTok{caption =} \StringTok{"第1种类型的统计表格样式"}
\NormalTok{) }\OperatorTok
\StringTok{  }\KeywordTok{kable_styling}\NormalTok{(}
    \DataTypeTok{latex_options =} \KeywordTok{c}\NormalTok{(}\StringTok{"striped"}\NormalTok{, }\StringTok{"hold_position"}\NormalTok{, }\StringTok{"repeat_header"}\NormalTok{),}
    \DataTypeTok{full_width =}\NormalTok{ F, }\DataTypeTok{position =} \StringTok{"center"}
\NormalTok{  ) }\OperatorTok
\StringTok{  }\KeywordTok{add_header_above}\NormalTok{(}\KeywordTok{c}\NormalTok{(}\StringTok{" "}\NormalTok{,}
    \StringTok{"$}\CharTok{\textbackslash{}\textbackslash{}\textbackslash{}\textbackslash{}}\StringTok{sigma^2$"}\NormalTok{ =}\StringTok{ }\DecValTok{2}\NormalTok{, }\StringTok{"$}\CharTok{\textbackslash{}\textbackslash{}\textbackslash{}\textbackslash{}}\StringTok{phi$"}\NormalTok{ =}\StringTok{ }\DecValTok{2}\NormalTok{,}
    \StringTok{"$}\CharTok{\textbackslash{}\textbackslash{}\textbackslash{}\textbackslash{}}\StringTok{tau^2$"}\NormalTok{ =}\StringTok{ }\DecValTok{2}\NormalTok{, }\StringTok{"$r=}\CharTok{\textbackslash{}\textbackslash{}\textbackslash{}\textbackslash{}}\StringTok{delta/}\CharTok{\textbackslash{}\textbackslash{}\textbackslash{}\textbackslash{}}\StringTok{phi$"}\NormalTok{ =}\StringTok{ }\DecValTok{1}
\NormalTok{  ), }\DataTypeTok{escape =}\NormalTok{ F) }\OperatorTok
\StringTok{  }\KeywordTok{footnote}\NormalTok{(}
    \DataTypeTok{general_title =} \StringTok{"注:"}\NormalTok{, }\DataTypeTok{title_format =} \StringTok{"italic"}\NormalTok{, }\DataTypeTok{threeparttable =}\NormalTok{ T,}
    \DataTypeTok{general =} \StringTok{"* 星号表示的内容很长很长很长很长很长长长长长长长长长长长长长长长长长"}
\NormalTok{  )}
\end{Highlighting}
\end{Shaded}

\rowcolors{2}{white}{gray!6}

\begin{ThreePartTable}
\begin{TableNotes}
\item \textit{注:} 
\item * 星号表示的内容很长很长很长很长很长长长长长长长长长长长长长长长长长
\end{TableNotes}
\begin{longtable}[t]{lrrrrrrr}
\caption{\label{tab:kable}第1种类型的统计表格样式}\\
\hiderowcolors
\toprule
\multicolumn{1}{c}{ } & \multicolumn{2}{c}{$\sigma^2$} & \multicolumn{2}{c}{$\phi$} & \multicolumn{2}{c}{$\tau^2$} & \multicolumn{1}{c}{$r=\delta/\phi$} \\
\cmidrule(l{2pt}r{2pt}){2-3} \cmidrule(l{2pt}r{2pt}){4-5} \cmidrule(l{2pt}r{2pt}){6-7} \cmidrule(l{2pt}r{2pt}){8-8}
Methods & Bias & RMSE & Bias & RMSE & Bias & RMSE & \\
\midrule
\endfirsthead
\caption[]{\label{tab:kable}第1种类型的统计表格样式 \textit{(continued)}}\\
\toprule
Methods & Bias & RMSE & Bias & RMSE & Bias & RMSE & \\
\midrule
\endhead
\
\endfoot
\bottomrule
\insertTableNotes
\endlastfoot
\showrowcolors
Mazda RX4 & 21.0 & 6 & 160.0 & 110 & 3.90 & 2.620 & 16.46\\
Mazda RX4 Wag & 21.0 & 6 & 160.0 & 110 & 3.90 & 2.875 & 17.02\\
Datsun 710 & 22.8 & 4 & 108.0 & 93 & 3.85 & 2.320 & 18.61\\
Hornet 4 Drive & 21.4 & 6 & 258.0 & 110 & 3.08 & 3.215 & 19.44\\
Hornet Sportabout & 18.7 & 8 & 360.0 & 175 & 3.15 & 3.440 & 17.02\\
\addlinespace
Valiant & 18.1 & 6 & 225.0 & 105 & 2.76 & 3.460 & 20.22\\
Duster 360 & 14.3 & 8 & 360.0 & 245 & 3.21 & 3.570 & 15.84\\
Merc 240D & 24.4 & 4 & 146.7 & 62 & 3.69 & 3.190 & 20.00\\
Merc 230 & 22.8 & 4 & 140.8 & 95 & 3.92 & 3.150 & 22.90\\
Merc 280 & 19.2 & 6 & 167.6 & 123 & 3.92 & 3.440 & 18.30\\
\addlinespace
Merc 280C & 17.8 & 6 & 167.6 & 123 & 3.92 & 3.440 & 18.90\\
Merc 450SE & 16.4 & 8 & 275.8 & 180 & 3.07 & 4.070 & 17.40\\
Merc 450SL & 17.3 & 8 & 275.8 & 180 & 3.07 & 3.730 & 17.60\\
Merc 450SLC & 15.2 & 8 & 275.8 & 180 & 3.07 & 3.780 & 18.00\\
Cadillac Fleetwood & 10.4 & 8 & 472.0 & 205 & 2.93 & 5.250 & 17.98\\
\addlinespace
Lincoln Continental & 10.4 & 8 & 460.0 & 215 & 3.00 & 5.424 & 17.82\\
Chrysler Imperial & 14.7 & 8 & 440.0 & 230 & 3.23 & 5.345 & 17.42\\
Fiat 128 & 32.4 & 4 & 78.7 & 66 & 4.08 & 2.200 & 19.47\\
Honda Civic & 30.4 & 4 & 75.7 & 52 & 4.93 & 1.615 & 18.52\\
Toyota Corolla & 33.9 & 4 & 71.1 & 65 & 4.22 & 1.835 & 19.90\\
\addlinespace
Toyota Corona & 21.5 & 4 & 120.1 & 97 & 3.70 & 2.465 & 20.01\\
Dodge Challenger & 15.5 & 8 & 318.0 & 150 & 2.76 & 3.520 & 16.87\\
AMC Javelin & 15.2 & 8 & 304.0 & 150 & 3.15 & 3.435 & 17.30\\
Camaro Z28 & 13.3 & 8 & 350.0 & 245 & 3.73 & 3.840 & 15.41\\
Pontiac Firebird & 19.2 & 8 & 400.0 & 175 & 3.08 & 3.845 & 17.05\\
\addlinespace
Fiat X1-9 & 27.3 & 4 & 79.0 & 66 & 4.08 & 1.935 & 18.90\\
Porsche 914-2 & 26.0 & 4 & 120.3 & 91 & 4.43 & 2.140 & 16.70\\
Lotus Europa & 30.4 & 4 & 95.1 & 113 & 3.77 & 1.513 & 16.90\\
Ford Pantera L & 15.8 & 8 & 351.0 & 264 & 4.22 & 3.170 & 14.50\\
Ferrari Dino & 19.7 & 6 & 145.0 & 175 & 3.62 & 2.770 & 15.50\\
\addlinespace
Maserati Bora & 15.0 & 8 & 301.0 & 335 & 3.54 & 3.570 & 14.60\\
Volvo 142E & 21.4 & 4 & 121.0 & 109 & 4.11 & 2.780 & 18.60\\*
\end{longtable}
\end{ThreePartTable}
\rowcolors{2}{white}{white}

\texttt{threeparttable\ =\ TRUE}
处理超长的注解标记,\texttt{add\_header\_above} 函数内的
\texttt{escape\ =\ F} 用来处理数学公式,\texttt{longtable\ =\ T}
表格很长时需要分页,因此使用续表

\begin{Shaded}
\begin{Highlighting}[]
\CommentTok{# 造一些数据}
\NormalTok{collapse_rows_dt <-}\StringTok{ }\KeywordTok{expand.grid}\NormalTok{(}
  \DataTypeTok{Country =} \KeywordTok{sprintf}\NormalTok{(}\StringTok{"Country with a long name %s"}\NormalTok{, }\KeywordTok{c}\NormalTok{(}\StringTok{"A"}\NormalTok{, }\StringTok{"B"}\NormalTok{)),}
  \DataTypeTok{State =} \KeywordTok{sprintf}\NormalTok{(}\StringTok{"State %s"}\NormalTok{, }\KeywordTok{c}\NormalTok{(}\StringTok{"a"}\NormalTok{, }\StringTok{"b"}\NormalTok{)),}
  \DataTypeTok{City =} \KeywordTok{sprintf}\NormalTok{(}\StringTok{"City %s"}\NormalTok{, }\KeywordTok{c}\NormalTok{(}\StringTok{"1"}\NormalTok{, }\StringTok{"2"}\NormalTok{)),}
  \DataTypeTok{District =} \KeywordTok{sprintf}\NormalTok{(}\StringTok{"District %s"}\NormalTok{, }\KeywordTok{c}\NormalTok{(}\StringTok{"1"}\NormalTok{, }\StringTok{"2"}\NormalTok{))}
\NormalTok{) }\OperatorTok
\StringTok{  }\KeywordTok{arrange}\NormalTok{(Country, State, City) }\OperatorTok
\StringTok{  }\KeywordTok{mutate_all}\NormalTok{(as.character) }\OperatorTok
\StringTok{  }\KeywordTok{mutate}\NormalTok{(}
    \DataTypeTok{C1 =} \KeywordTok{rnorm}\NormalTok{(}\KeywordTok{n}\NormalTok{()),}
    \DataTypeTok{C2 =} \KeywordTok{rnorm}\NormalTok{(}\KeywordTok{n}\NormalTok{())}
\NormalTok{  )}
\NormalTok{row_group_label_fonts <-}\StringTok{ }\KeywordTok{list}\NormalTok{(}
  \KeywordTok{list}\NormalTok{(}\DataTypeTok{bold =}\NormalTok{ T, }\DataTypeTok{italic =}\NormalTok{ T),}
  \KeywordTok{list}\NormalTok{(}\DataTypeTok{bold =}\NormalTok{ F, }\DataTypeTok{italic =}\NormalTok{ F)}
\NormalTok{)}

\KeywordTok{kable}\NormalTok{(collapse_rows_dt, }\StringTok{"latex"}\NormalTok{, }\DataTypeTok{longtable =} \OtherTok{TRUE}\NormalTok{,}
  \DataTypeTok{booktabs =}\NormalTok{ T, }\DataTypeTok{align =} \StringTok{"c"}\NormalTok{, }\DataTypeTok{linesep =} \StringTok{""}\NormalTok{,}
  \DataTypeTok{caption =} \StringTok{"第2种类型的统计表格样式"}
\NormalTok{) }\OperatorTok
\StringTok{  }\KeywordTok{kable_styling}\NormalTok{(}
    \DataTypeTok{latex_options =} \KeywordTok{c}\NormalTok{(}\StringTok{"striped"}\NormalTok{, }\StringTok{"hold_position"}\NormalTok{, }\StringTok{"repeat_header"}\NormalTok{),}
    \DataTypeTok{full_width =}\NormalTok{ F, }\DataTypeTok{position =} \StringTok{"center"}
\NormalTok{  ) }\OperatorTok\StringTok{ }
\StringTok{  }\KeywordTok{column_spec}\NormalTok{(}\DecValTok{1}\NormalTok{, }\DataTypeTok{bold =}\NormalTok{ T) }\OperatorTok
\StringTok{  }\KeywordTok{collapse_rows}\NormalTok{(}\DecValTok{1}\OperatorTok{:}\DecValTok{3}\NormalTok{,}
    \DataTypeTok{latex_hline =} \StringTok{"custom"}\NormalTok{, }\DataTypeTok{custom_latex_hline =} \DecValTok{1}\OperatorTok{:}\DecValTok{3}\NormalTok{,}
    \DataTypeTok{row_group_label_position =} \StringTok{"stack"}\NormalTok{,}
    \DataTypeTok{row_group_label_fonts =}\NormalTok{ row_group_label_fonts}
\NormalTok{  )}
\end{Highlighting}
\end{Shaded}

\rowcolors{2}{white}{gray!6}

\begin{longtable}[t]{>{\bfseries}cccccc}
\caption{\label{tab:kable-comp1}第2种类型的统计表格样式}\\
\hiderowcolors
\toprule
 &  & City & District & C1 & C2\\
\midrule
\endfirsthead
\caption[]{\label{tab:kable-comp1}第2种类型的统计表格样式 \textit{(continued)}}\\
\toprule
Country & State & City & District & C1 & C2\\
\midrule
\endhead
\
\endfoot
\bottomrule
\endlastfoot
\showrowcolors
\addlinespace[0.3em]
\multicolumn{6}{l}{\textit{\textbf{Country with a long name A}}}\\
\addlinespace[0.3em]
\multicolumn{6}{l}{State a}\\
\hspace{1em}\hspace{1em} &  & City 1 & District 1 & 0.8105812 & -1.5852928\\

\hspace{1em}\hspace{1em} &  &  & District 2 & 1.4179038 & 0.6183236\\
\cmidrule{3-6}
\hspace{1em}\hspace{1em} &  & City 2 & District 1 & 1.3476411 & -0.0471493\\

\hspace{1em}\hspace{1em} &  &  & District 2 & -0.5016625 & -0.2134680\\
\cmidrule{2-6}
\addlinespace[0.3em]
\multicolumn{6}{l}{State b}\\
\hspace{1em}\hspace{1em} &  & City 1 & District 1 & -0.2870107 & -0.4970279\\

\hspace{1em}\hspace{1em} &  &  & District 2 & -0.9662371 & 0.5951959\\
\cmidrule{3-6}
\hspace{1em}\hspace{1em} &  & City 2 & District 1 & 0.3049904 & -0.1769102\\

\hspace{1em}\hspace{1em} &  &  & District 2 & -0.2219533 & -0.3058346\\
\cmidrule{1-6}
\addlinespace[0.3em]
\multicolumn{6}{l}{\textit{\textbf{Country with a long name B}}}\\
\addlinespace[0.3em]
\multicolumn{6}{l}{State a}\\
\hspace{1em}\hspace{1em} &  & City 1 & District 1 & 1.1572648 & 1.0044060\\

\hspace{1em}\hspace{1em} &  &  & District 2 & 0.1663139 & 1.5270802\\
\cmidrule{3-6}
\hspace{1em}\hspace{1em} &  & City 2 & District 1 & -0.8831136 & 1.2107975\\

\hspace{1em}\hspace{1em} &  &  & District 2 & -0.5066262 & -0.1710977\\
\cmidrule{2-6}
\addlinespace[0.3em]
\multicolumn{6}{l}{State b}\\
\hspace{1em}\hspace{1em} &  & City 1 & District 1 & 0.3320892 & -1.2652650\\

\hspace{1em}\hspace{1em} &  &  & District 2 & -0.7597824 & 0.6592363\\
\cmidrule{3-6}
\hspace{1em}\hspace{1em} &  & City 2 & District 1 & -1.4505810 & -1.9158731\\

\hspace{1em}\hspace{1em} &  &  & District 2 & 1.5479971 & -0.9852655\\*
\end{longtable}
\rowcolors{2}{white}{white}

To compile this example to PDF, you need XeLaTeX. You are recommended to
install TinyTeX (which includes XeLaTeX):
\url{https://yihui.name/tinytex/}.

软件信息

\begin{Shaded}
\begin{Highlighting}[]
\NormalTok{xfun}\OperatorTok{::}\KeywordTok{session_info}\NormalTok{(}\DataTypeTok{packages =} \KeywordTok{c}\NormalTok{(}\StringTok{"rmarkdown"}\NormalTok{,}\StringTok{"bookdown"}\NormalTok{,}\StringTok{"kableExtra"}\NormalTok{),}
                   \DataTypeTok{dependencies =} \OtherTok{FALSE}\NormalTok{)}
\end{Highlighting}
\end{Shaded}

\begin{verbatim}
## R version 3.5.0 (2017-01-27)
## Platform: x86_64-pc-linux-gnu (64-bit)
## Running under: Ubuntu 14.04.5 LTS
## 
## Locale:
##   LC_CTYPE=en_US.UTF-8       LC_NUMERIC=C              
##   LC_TIME=en_US.UTF-8        LC_COLLATE=en_US.UTF-8    
##   LC_MONETARY=en_US.UTF-8    LC_MESSAGES=en_US.UTF-8   
##   LC_PAPER=en_US.UTF-8       LC_NAME=C                 
##   LC_ADDRESS=C               LC_TELEPHONE=C            
##   LC_MEASUREMENT=en_US.UTF-8 LC_IDENTIFICATION=C       
## 
## Package version:
##   bookdown_0.7.18  kableExtra_0.9.0 rmarkdown_1.10
\end{verbatim}

\bibliography{book.bib,packages.bib}


\end{document}
